\documentclass[12pt, a4paper, oneside]{article}
\usepackage[utf8]{inputenc}
\usepackage{graphicx}
\usepackage[ngerman]{babel}
\usepackage{amssymb}
\usepackage{appendix}
\usepackage{listings}
\lstset{language=c++}
\lstset{
    tabsize=2,
    numbers=left
}
\usepackage{hyperref}
\hypersetup{
    colorlinks,
    linkcolor=black,
    urlcolor=blue,
    citecolor=black
}
% to add references
\usepackage{biblatex}
\addbibresource{ref.bib}
% to prevent warning
\usepackage{csquotes}

%to prevent graphics from floating ([H])
\usepackage{float}
% Table colors
\usepackage[table,xcdraw]{xcolor}
\usepackage{booktabs}
% Tims Zeug
\usepackage{contour}
\usepackage{ulem}
\renewcommand{\ULdepth}{1.8pt}
\contourlength{0.8pt}
\newcommand{\myuline}[1]{%
  \uline{\phantom{#1}}%
  \llap{\contour{white}{#1}}%
}

\begin{document}

% Titelseite
\begin{titlepage}
\begin{center}
\includegraphics[width=0.5\textwidth]{img/lmg8.png}\\[1.5cm]
\textsc{\Large Facharbeit im Fach Informatik }\\[0.4cm]
\textsc{Leitung: Frau Müller}\\[0.4cm]
\hrule
\bigskip
\huge Die Entwicklung eines Compilers zu einem auf Java basierenden Befehlssatz\\[0.5cm]
\hrule
\begin{minipage}{0.4\textwidth}
\begin{flushleft} \large
\end{flushleft}
\end{minipage}
\hfill
\begin{minipage}{0.4\textwidth}
\begin{flushright} \large
\bigskip
\medskip
Tobias Palzer \\
MSS 11 \\
Stammkurs M1  \\[0.8cm]
\end{flushright}
\end{minipage}
\vfill
{\large \today}
\end{center}
\end{titlepage}

% Inhaltsverzeichnis
\pagenumbering{roman}
\tableofcontents
\newpage
\listoffigures
\newpage

\pagenumbering{arabic}

\section{Lastenheft}
\textbf{UR001}\\
\textbf{Aussage} TODO.\\
\textbf{Priorität A} \\
\newline
\textbf{UR001}\\
\textbf{Aussage} TODO.\\
\textbf{Priorität A} \\

\newpage
\section{Pflichtenheft}

\newpage
\section{Dokumentation}
\myuline{23.09.2019:} \\
Tobias hat eine Vorlage für ein Eclipse-Projekt erstellt, die als Grundlage für das Softwareprojekt dient.
Diese kann über den \glqq git clone\grqq{}-Befehl heruntergeladen werden und funktioniert über Eclipse.
Außerdem enthält die Vorlage die Grafikbefehle, die als Grundlage für die grafische Oberfläche dienen sollen.


\end{document}
